\documentclass{article}

\usepackage{tikz}
\usepackage{bm}
\usetikzlibrary{calc}

\begin{document}

\def\layersep{-2.5cm}

\tikzset{neuron/.style={circle,thick,fill=black!25,minimum size=35pt,inner sep=0pt},
    yhat neuron/.style={neuron,fill=white,draw},
    h neuron/.style={neuron,fill=white,draw},
    x neuron/.style={neuron,fill=white,draw},
    hoz/.style={rotate=-90}}   %<--- for labels

\begin{tikzpicture}[-,draw=black, node distance=\layersep,transform shape,rotate=90]  %<-- rotate the NN

% Draw the "hat(y_t)" layer nodes
\foreach \name / \y in {3/-2,6/-1,9/}
\node[yhat neuron, hoz] (YHAT-\name) at (0,-\name) {\color{red}$\hat{y}^{(t\y)}$};

%\node[hoz] (N-4) at (0,-4) {$\dots$};


% Draw the "h" layer nodes
\foreach \name / \y in {0/-3,3/-2,6/-1,9/, 12/+1}
\path[yshift=0cm] node [h neuron, hoz] (H-\name) at (\layersep,-\name cm) {\color{red}$h^{(t\y)}$};


% Draw the input layer nodes:
\foreach \name / \y in {3/-2,6/-1,9/}
\path[yshift=0cm] node [x neuron, hoz] (X-\name) at (\layersep*2,-\name) {\color{red}$\bm{x}^{(t\y)}$};


% Connect middle nodes in the "h" layer vertically with yhats.
    \foreach \source in {3,6,9}
    		\path (H-\source.north) edge[->,line width=1] (YHAT-\source.south);
            
    
% Connect middle nodes to each other.
    \path (H-0.east) edge[->,line width=1] (H-3.west);
    \path (H-3.east) edge[->,line width=1] (H-6.west);
    \path (H-6.east) edge[->,line width=1] (H-9.west);
    \path (H-9.east) edge[->,line width=1] (H-12.west);


% Connect input nodes vertically with the "h" layer.
    \foreach \source in {3,6,9}
    		\path (X-\source.north) edge[->,line width=1] (H-\source.south);
     
               

\end{tikzpicture}
\end{document}